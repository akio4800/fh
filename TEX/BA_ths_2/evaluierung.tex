\chapter{Evaluierung}
\section{Einleitung}
Das Outlookconnectormodul ist im MRP-System teilintegriert. Das Modul befand sich im gleichen Repository, wurde aber nicht vollständig integriert, da dieses Modul in einem eigenen Zweig des Repositories verwaltet wurde. Dieser Umstand führte dazu, dass das Modul kein Teil der vordefinierten Testumgebung darstellte. 
\section{Testumgebung}
Die Testumgebung wurde mit Hilfe eines \textit{Jenkins-Maven-Testserver} realisiert, welcher selbständiges automatisiertes Testen zulässt. In dem Repositories gab es zwei Zweige, welche \textit{master} und \textit{dev} genannt werden. Nur diese zwei wurden in die automatisierte Testumgebung integriert und daher in fixierten Zeitabständen getestet. Da das zu bearbeitende Modul kein Teil der zwei Zweige war, konnte es nicht getestet werden. Durch diese Aufteilung wurden bereits fertiggestellte Module verspätet getestet. Das zu bearbeitende Modul wurde erst nach Beendigung des Praktikums in das Testframework integriert. Durch den Umstand des langsamen Testprozesses musste auf andere Testmittel umgestiegen werden.
\section{Benutzerinteraktives Testen}
Die gewählte Methodik wird als Benutzer interaktives Testen bezeichnet, besser bekannt als \textit{Whitebox-Test}. Diese Methodik beruht auf der visuellen Kontrolle des Benutzers. Dabei überprüft der Programmierer die Ergebnisse in der grafischen Oberfläche bzw. Konsole. Diese Methodik kann nicht als Test im Sinne eines Testframeworks angesehen werden, da nicht überprüft werden kann, ob die Testergebnisse den erwarteten Ergebnissen entsprechen. Dies kann erst mit der späteren Integration des Moduls überprüft werden. Diese Testmethodik dient rein zur visuellen Kontrolle ob die geforderten Datenfelder richtig in der Konsole angezeigt werden. Da explizite Tests fehlen, ist relativ schwer zu kontrollieren, ob diese durchgehen würden. Trotzdem können mit Hilfe dieser Methodik alle Funktionen eines Programms getestet werden, auch wenn kein geeignetes Testframework vorhanden ist. Da das Modul mit einen Server kommunizieren muss, ist automatisiertes Testen mit Hilfe eines Frameworks relativ kompliziert und damit unpraktikabel. \\

Folgende Funktionen wurden  im Rahmen der Arbeit überprüft:\\

\begin{itemize}
	\item E-Mails senden und empfangen
	\item Kontakte anlegen und löschen
	\item Termine anlegen, anzeigen und löschen
	\item Verfügbarkeiten abrufen
	\item Filter anlegen, bearbeiten und löschen \\
\end{itemize}

Alle obengenannten Funktionen wurden ausführlich getestet und ergaben schlussendlich keinen Fehler.
