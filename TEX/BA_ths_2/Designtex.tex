\chapter{Design}
Im Prinzip gibt es 3 verschiedene Konzepte zum gegenständlichen Thema:
\section{Konzept 1}
Manuelle Interaktion mit der Outlookoberfläche: Der User verwendet das Outlook an u. hat dabei die Möglichkeit folgende Funktionen zu verwenden:
Microsoft Exchange Server ist eine Groupware- und E-Mail-Transport-Server-Software des Unternehmens Microsoft. Sie dient der zentralen Ablage und Verwaltung von E-Mails, Terminen, Kontakten, Aufgaben und weiteren Elementen für mehrere Benutzer und ermöglicht so die Zusammenarbeit in einer Arbeitsgruppe oder in einem Unternehmen. Exchange Server setzt eine Microsoft-Windows-Server-Software voraus und findet deshalb vor allem in von Microsoft-Produkten geprägten Infrastrukturen Verwendung.
Um als Anwender die Funktionen von Exchange Server nutzen zu können, ist eine zusätzliche Client-Software nötig. Microsoft liefert dafür das separate Programm Microsoft Outlook, in Exchange Server selbst ist bereits die Webanwendung Outlook Web App für den Browser-Zugriff enthalten.
Im Rahmen des Microsoft-Online-Dienstes Office 365 kann d}ie Server-Software unter dem Titel Exchange Online auch bzw. zusätzlich bei Microsoft („Cloud-Computing“) statt im eigenen Unternehmen („On Premise“) betrieben werden, was vor allem für kleinere Unternehmen ohne ausgeprägte IT-Infrastruktur interessant sein kann.
E-Mails:
E-Mails können gesendet, empfangen u. in diversen Ordnern hinterlegt werden. Neue E-Mails können Termine enthalten, welche als Termineinladungen gelten u. nach der Bestätigung in den Kalender übertragen werden. Weiters ist eine schnelle Abfrage der im persönlichen Ordner gespeicherten E-Mails durch eine ausgeklügelte Suchfunktion möglich.
Termine u. Kalender
Jeder Benutzer verfügt mindestens über einen Kalender. Dieser Kalender besitzt alle persönlichen Termine sowie das firmenmäßige Umwelt. Weiters werden zusätzliche Termine wie Feiertage, Geburtstage etc. angezeigt. Eine Option ist die Übermittlung von Texten bei den Ein-/Ausladungen.
Darüber hinaus können ganze Gruppen eingeladen werden. Eine Errinnergungsfunktion wir als eine weitere Servicefunktion angeboten.
Aufgaben
Aufgaben sind abzuarbeiten u. als erledigt zu markierenls Empfänger verwendet werden.
Kontakte / Adressen
Kontakte können erstellt werde wenn eine E-Mailadresse bzw. eine Telefonnummer vorhanden ist. Kontakte sind in Gruppen unterteilt u. können als Gruppe einen Termin zugewiesen werden. Eine Gruppe kann. Kontakte enthalten weiters noch weitere Information der jeweiligen Person. 
Globales Adressbuch
Alle Kontakte sind nach Gruppen geordnet u. eingeteilt. Diesbezüglich sind Such-bzw. Filterfunktion verfügbar.
Outlook Web App, Webzugriff auf die Funktionen des Servers

Es gibt Webseiten, wo ohne Outlookprogramm auf das Outlookkonto zugegriffen kann. Das Webinterface hat die gleiche Funktionalität wie das Outlook selbst. Das Java EWS funktioniert nach dem gleichen Prinzip. Das EWS hat die gleiche Funktionalität wie die Webschnittstelle. Es gibt eine mobile Outlookversion, die Mithilfe der Webschnittschnittstelle auf den Outlookserver bzw. Outlook zugreift. Die weist eine Funktionalitätensammlung auf.
Konzept 2
Ansteuerung von Outlook mit Hilfe von EWS
Der Benutzer verfügt über Programmierkenntnisse u. kann die Schnittstelle ansteuern u. verwenden.
Die EWS API wird von Microsoft zur Verfügung gestellt u. dient zur Ansteuerung von Outlook über eine geeignete Programmiersprache. Diese API beinhaltet dieselbe Funktionalität wie die Webschnittstelle u. arbeitet nach denselben Prinzip der Webschnittstellen. EWS bietet zusätzliche Funktionen zu der Webschnittstelle wie abfragen von Verfügbarkeiten u. Benachrichtigungen. Die EWS API ermöglicht den Programmieren eine einfachere Ansteuerung von Outlookservern. Die Schnittstelle bietet eine ausreichende Funktionalität sodass im Outlookclient keine Aktionen dd
Konzept 3
Alle Funktionalitäten, die das MRP-System bereitstellt sind vorhanden  bzw. in Planung.
Das MRP-Modul, welches von CGM bereitgestellt wurde, beinhaltet folgende Funktionalitäten:
1.	Zu Beginn müssen die verfügbaren Ressourcen angelegt u. zugeordnet werden.  
2.	Zuordnung der ressourcenbezogenen Daten: Jeder  Ressource werden Informationen beigegeben
3.	Aus den Ressourcen wird mithilfe von Kapazitäten Einheiten gebildet.
4.	Aus diesen Einheiten werden Vorschläge gebildet. Diese Vorschläge beinhalten lediglich einer …. eine zeitliche Spanne. Um einen Termin festlegen zu können müssen Ressourcen zugeordnet werden. Dies entsteht aus der bereits hinterlegten Personendaten sowie aus den Verfügbarkeitsabfragen der EWS Schnittstelle. Mit diesen erweiterten Funktionen können konkrete Termine generiert u. nach deren Bestätigung ins Outlook exportiert werden. Die exportierten Termine können im betreffenden Kalender eingesehen werden. Das Anlegen von Ressourcen u. das Selektieren von exportierten Terminen erfolgt manuell.

