\chapter {Zusammenfassung}
In diesem Kapitel werden die Ergebnisse zusammengefasst und diskutiert.
\section{Resultate}
Bei dieser Arbeit wurde ein Programm implementiert, welches als Modul für das MRP-System der Firma CGM fungiert. Die Implementierung wurde mittels JAVA durchgeführt. Es wurde die Java EWS API verwendet. Die EWS API dient zur Ansteuerung eines Microsoft Outlookservers mithilfe von Programmmethoden. Das Modul verwendete diese API um bereitgestellte Termine aus dem MRP-System an das Microsoft Outlook zu exportieren. Die Daten wurden von weiteren Modulen des MRP-Systems generiert und bereitgestellt. Allerdings fehlte die Integration des Moduls in das gesamtheitliche MRP-System, da dieses zum damaligen Zeitpunkt noch nicht fertiggestellt war. Das Testen erfolgte durch benutzerinteraktives Testen. Dieses beruhte auf einer rein visuellen Betrachtung des Benutzers. Die diesbezüglichen Tests verliefen positiv.
\section{Diskussion}
Das in dieser Arbeit erstellte Programmmodul ist universell einsetzbar, wurde aber geringfügig an das MRP-System angepasst. Die EWS API mit ihren komplexen Datentypen wurden durch das vereinfachte DTO ersetzt. Die universelle Einsetzbarkeit wurde damit ermöglicht. Die Implementierung der verwendeten Filter wurde generisch gehalten, um eine universelle Einsetzbarkeit zu ermöglichen.
