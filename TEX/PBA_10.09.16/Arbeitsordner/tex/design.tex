\chapter{Konzept und Design}
In diesem Kapitel wird das Konzept zur Übertragung von Daten aus dem MRP ins Microsoft Outlook beschrieben. 


\section{Ansteuerung mittels MRP}
MRP ist eine von CGM entwickelte Software, die Termine für Operationen in Krankenhäusern unter der Berücksichtigung aller beteiligter Ressourcen optimiert und berechnet. Das MRP-Modul beinhaltet folgende Funktionalitäten\cite{html5,g3r,g3mrp}:\\
\begin{enumerate}
	\item Zu Beginn werden die verfügbaren Ressourcen (Personen, Räume, Geräte, Betten, Materialien) angelegt und zugeordnet.  
	\item Zuordnung der ressourcenbezogenen Daten: Jeder  Ressource werden Informationen zugeteilt.
	\item Mittels der Ressourcen werden mithilfe von Kapazitäten(terminliche Verfügbarkeiten) Organisationseinheiten (Zeitbereiche mit verknüpften Ressourcen) gebildet.
	\item Aus diesen Einheiten werden Vorschläge gebildet. Diese Vorschläge beinhalten lediglich eine Zeitspanne. Um einen Termin festlegen zu können, müssen Ressourcen  definiert werden. Dies entsteht aus den bereits hinterlegten Personendaten sowie aus den Verfügbarkeitsabfragen der EWS Schnittstelle. Mit diesen erweiterten Funktionen können konkrete Termine generiert u. nach deren Bestätigung ins Outlook exportiert werden. Die exportierten Termine können im betreffenden Kalender eingesehen werden. Das Anlegen von Ressourcen u. das Selektieren von exportierten Terminen erfolgt manuell.\\
\end{enumerate}	

Die Erstellung eines MRP Moduls erfolgt nach den Programmierrichtlinien von CGM. Dieses Modul dient zum Export von generierten Terminen ins Outlook. Weiters werden Filter- und Suchfunktionen bereitgestellt. Diese können MRP-Appointments und Outlook-Appointments filtern und suchen.




