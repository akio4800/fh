\chapter{Evaluierung}
\section{Einleitung}
Das Outlookmodul ist im MRP-System teilintegriert. Die Testumgebung wird mit Hilfe eines Jenkins-Maven-Testserver realisiert, welcher selbständiges automatisiertes Testen zulässt. Bei einen Jenkins-Maven-Testserver handelt es sich um einen Linux-Server, welcher das gesamte Projekt in regelmäßigen Abständen compiliert und die vorgesehenen Tests ausführt. Die Ergebnisse sind in einer Weboberfläche einsehbar. In dem Repository gibt es die Branches \textit{master} und \textit{dev}. Diese werden in die automatisierte Testumgebung integriert und in fixierten Zeitabständen getestet. Da das zu bearbeitende Modul kein Teil der zwei Branches ist, kann es nicht getestet werden. Durch den Umstand des Testprozesses muss auf andere Testmittel umgestiegen werden.

\section{Benutzerinteraktives Testen}
Auf Grund des Testprozesses wird auf die Methodik benutzerinteraktives Testen zurückgegriffen. 
Die gewählte Methodik ist besser bekannt als \textit{Whitebox-Test}.\cite{SpillnerLinz05} Diese Methodik beruht auf der visuellen Kontrolle des Programmierers. Dabei überprüft der Programmierer die Ergebnisse in der grafischen Oberfläche bzw. Konsole. Die Testmethodik dient rein zur visuellen Kontrolle ob die geforderten Datenfelder richtig in der Konsole angezeigt werden. Trotzdem kann mit Hilfe dieser Methodik die Funktion des Programms getestet werden. Dies ist auch möglich wenn kein geeignetes Testframework vorhanden ist. Da das Modul mit einem Server  kommuniziert, bei dem die Verarbeitung der Anfragen einen Engpass darstellt, stellt sich automatisiertes Testen mittels Frameworks als aufwendig und komplex dar. 
\newpage
Im Rahmen der Testung werden folgende Funktionen überprüft:\\

\begin{table}[H]
\centering
\begin{tabular}{|l|c|}
\hline
Getestete Funktionen&Tests erfolgreich\\ \hline
Appointments anlegen& 16\\ \hline
Appointments abrufen& 24\\ \hline
Availablilities abfragen&20\\ \hline
Kalender anlegen&21\\ \hline
Kalender abfragen&21\\ \hline
Kontakte abrufen&14\\ \hline
Adressbuch abfragen&30\\ \hline
Emails senden     &40\\ \hline
Emails abrufen     &45\\ \hline
Serververbindung initialisieren& 5\\ \hline
Serververbindung speichern & 5 \\ \hline
Serververbindung abfragen     &    10    \\ \hline

Filter verwalten     &       38 \\ \hline
Filterkomponenten verwalten     &    64    \\ \hline
Filterkomponenten zuordnen     &    23    \\ \hline
MRP-Filter verwalten     &   12     \\ \hline
MRP-Filterkomponenten verwalten     &       17 \\ \hline
MRP-Appointments exportieren & 0 (keine Tests vorhanden) \\ \hline
\end{tabular}
\caption{Tests}
\end{table}

